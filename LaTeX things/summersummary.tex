\documentclass{article} 
\usepackage{xcolor} 
\usepackage[inline]{enumitem}
\usepackage{hyperref}
\usepackage{sectsty}
\sectionfont{\fontsize{12}{15}\selectfont}
\hypersetup{
    colorlinks=true,
    linkcolor=blue,
    filecolor=magenta,      
    urlcolor=cyan,
}
\title{Summer 2017 Research Revisit} 
\author{Jonathan Mercedes Feliz} 
\begin{document}
\maketitle{}
\section{Week 1 (5/30 - 6/2):}
\
\par First week of the summer consisted of taking the computational bootcamp from 9 to 5. During this bootcamp I learned how to use bash in terminal, navigating and using shorcut commands on the keyboard. Also learned to use and navigate git; creating repositories, forking repositories, pushing and pulling commits, and learning what version control is. Python was another important component of the bootcamp that I learned. Using python and its libraries in either terminal or jupyter notebook. While I had previously dabbled with python I got out much more and found it easier to navigate and understand the libraries to make them do what I wanted to.
\section{Week 2 (6/5 - 6/9):}
\
\par  This week met and discussed with mentors about summer project and inital readings to begin. First two days of that week there was a galaxy simulation workshop working with introductory notebooks in jupyter notebook. Within the workshop we looked at various programs and simulations, looking at the information they gave us about galaxies, stars and other various properties. During this workshop they gave us some more work and practice with numpy and matplotlib libraries. This really helped and it heightened my experience with it and helped me get used to it more. Around the end of the workshop we were tasked to determine the difference between spiral and elliptical galaxies using the Illustris simulation and website. By defining functions and plotting the parameters we thought would differentiate between the two we tried to classify a list of galaxies to one of the two. 
\
I was given two articles to read and a section of a book. One article was an observational paper about a massive black hole in a dwarf galaxy. The other was about simulations of merging galaxies with black holes. That same Friday I learned from previous cohorts on how to give an APOD presentation for my presentation  the following week.
\section{Week 3 (6/12 - 6/16):}
\
\par  During this week most of my time was devoted to writing summaries of the papers, and working on my APOD. Whilst also getting work done with text files plotting the information they carry. First plot was a small file of BH mass vs time. Later on received some files on a dwarf galaxy simulation with no merger by the end of the week.
\section{Week 4 (6/19 - 6/23):}
\
\par  Continuing on from the previous week I plotted the information of the text files. While doing the plots, met with my mentors discussing my progress and the summaries I completed the previous week. While also doing some exercises in python using the previous data used in the text files to find the time with max accretion rate and by using that comparing it to the average and looking at the same thing with star formation rate.
\section{Week 5 (6/26 - 6/30):}
\
\par  During this week I received raw ouput files from the simulation (merger and no merger). Used these files and created functions that made subplots that showed SFR, BH mass, BH mass accretion, and BH mass accretion due to Eddington fraction with respect to time. While doing this for both merger and no merger, I did the same process in looking at t${_max}$ like the previous week. After completing this we looked into finding the Luminosity of the black hole. So looking at chapter 4 of the Maoz book \textit{Astrophysics in a Nutshell} I got the formula for Luminosity in terms of mass accretion and the Eddington Luminosity. Also converted the units from M${_\odot}$ and Gyr to ergs and seconds, while also converting from internal units to proper units for the mass and time. After getting the Luminosity for both simulation cases and plotting them we looked through the ADS database to find the conversion from bolometric luminosity to the X-ray luminosity using a paper by RuNoe 2012 and then plotted that with time. Relating with the observational paper showing that the black hole in a dwarf galaxy is possible. During this week I was given a link to the directory with the full simulation files. From there I looked into the very first snapshot and just fiddled at what's in it. By the end of the week I created a function to get the center of mass for all particles in a certain particle group.
\section{Week 6 (7/3 - 7/7):}
\
\par  
Mastered for loop in python to create text file with data pertaining to the calculation of center of mass for all snapshots including the Stars and New Stars Particles. Also finished the summary/revisit of what I did throughout the sumnmer since I started. Worked on the presentation of my summer research project for this coming Friday.
\end{document}