\documentclass{article} 
\usepackage{xcolor} 
\usepackage[inline]{enumitem}
\usepackage{hyperref}
\hypersetup{
    colorlinks=true,
    linkcolor=blue,
    filecolor=magenta,      
    urlcolor=cyan,
}
\title{BHs in Galaxy Mergers Growth and Activity Summary ${\&}$ Thought Process} 
\author{Jonathan Mercedes Feliz} 
\begin{document} 
\maketitle{}
\section{Summary} 

\
\par  Looking at and studying supermassive black holes in merging galaxies using\colorbox{yellow}{hydrodynamical simulations} with high spatial and temporal resolution. These simulations vary in initial mass ratio,\colorbox{yellow}{orbital configuration}, and \colorbox{yellow}{the gas fraction}. These simulations help with multiple revelations. 
\begin{enumerate*}[label=(\roman*),%before=\unskip{ i.e., },
itemjoin={{, }}, itemjoin*={{, and }}]
\item Like addressing when and why during a merger, increased BH accretion occurs, quantifying\colorbox{yellow}{gas inflows} and BH accretion rates \item As well as looking at the relative effectiveness in inducing AGN activity of merger-related versus\colorbox{yellow}{secular}-related causes by looking at different steps of the interaction \item observing and finding out which galaxy merger tends to enhance BH accretion, after seeing that the initial mass ratio is the most important factor \item We look at the evolution of the BH masses, finding that the\colorbox{yellow}{BH mass contrast} tends to decrease in minor mergers and increase in major ones. Hinting in the existence of a preferential range of mass ratios for BHs in the last pairing stages \item in both merging and\colorbox{yellow}{dynamically quiescent} galaxies, the gas accreted by the BH is not neccesarily the gas with low angular momentum but the one that loses angular momentum. \end{enumerate*} 
\
\par  These BHs are believed to be in the center of many galaxies in the universe, as a result of this, a fraction of these BH-galaxy pairs result in AGNs, active galactic nucleus, a result of high level accretion of gas onto the BH in the center. The gas starts to lose angular momentum and flows toward the center, where it gets accreted and drives the growth of BHs. Possible explanations have arisen, which includes minor interactions, which could include minor mergers, and/or secular processes , which can include instabilities driven by\colorbox{yellow}{bars and violent gas instabilities at high redshift}. Not all AGN activity is merger driven though. By performing\colorbox{yellow}{kinematic-neighbour studies} and morphological studies of a sample of AGN hosts and a sample of inactive galaxies, you find that strong interactions are not more common among AGN than with normal galaxies, while also finding similar\colorbox{yellow}{distortion fractions} between active and inactive galaxies. While there also suggestions that support mergers increase AGN activity with other studies conducted by others. This shifts the questioning from "Is AGN activity merger driven?" to "Which galaxy mergers enhance AGN activity?". Understanding if, when and how mergers trigger AGN activity, depending on the dynamics and thermodynamics of said merger.\colorbox{yellow}{1:1 mergers}, while triggering the strongest burst of activity, are extremely rare mergers in the Universe. Observing different mass-ratios allow for a more extensive look into what happens to a BH and its activity. This presents detailed analysis aimed at understanding which BH tends to be more active, or more visible during a merger event. Anticipating that the interaction between both galaxies must redistribute the angular momentum of the gas in order to drive\colorbox{yellow}{consistent inflows} triggering any AGN activity. In minor mergers, the secondary galaxy is significantly affected by the\colorbox{yellow}{gravitational torques} exerted by the primary, while the primary galaxy basically remains\colorbox{yellow}{unpertubated} throughout the whole interaction., major  WHile on the other and, major mergers significantly affects both galaxies, triggering major accretion episodes on both BHs. Within these merger simulations the two disc galaxies set at z = 3(near\colorbox{yellow}{the peak of the cosmic merger rate}), with different mass ratios, orbital configuration and gas fractions.
\par   Choosing a\colorbox{yellow}{1:4 coplanar, prograde-prograde merger} because the mass ratio \textit{q}${_G}$ = 0.25 is the median of the mass ratios decided. While minor mergers are less than \textit{q}${_G}$ and major mergers are more than \textit{q}${_G}$. The two BHs within their respective galaxies start off at a distance of 74 kpc. Throughout the simulation the figures show varying shifts between different galaxies around both BHs. Quantity relations like separation between the BHs, BH accretion rate and BH Eddington accretion rate, star formation rate across both galaxies, and SFR in concentric spheres around the BH, gas mass in concentric spheres around the center of mass near the BH, and gas\colorbox{yellow}{specific angular momentum} magnitude around the concentric shells around the center of mass around the BH. The time scales for BH merging are still very uncertain. So assuming a specific time scale for the merging would be highly arbitrary. During the final stages of the encounter, the BH accretion rate becomes\colorbox{yellow}{quasi-periodic}. During the second pericentric passage, central SFR around the secondary BH can increase by more than three\colorbox{yellow}{orders of magnitude} from its previous levels and makes up for about the total SFR in the system. During the final stages the two BHs have higher SFR than in the first stage, while SFR around the primary BH is more centralized.
\par   During the merger stage,\colorbox{yellow}{merger dynamics} trigger loss of gas angular momentum (and gas inflows), linked to bursts of high SFR and BH accretion rate. As the merging encounter ends, the remnant galaxy shows with SFR and BH accretion rate return to levels in the stochastic stage. BH accretion is a gas limited process; without it being in the vicinity of the BH there would be no accretion. Another important condition for accretion is that the specific angular momentum of the central gas needs to be low. With the high resolution of the simulations you're shown that what matters isn't the amount of specific angular momentum of the gas, but its\colorbox{yellow}{temporal gradient}. The gas that gets accreted is the one that loses angular momentum, any decrease of specific angular momentum, regardless of the value is enough to cause an inflow of gas towards the center and enhances accretion onto the BH. Throughout all three stages, stochastic, merger and remnant, BH accretion rate and central gas specific angular momentuma are strongly\colorbox{yellow}{anti-correlated}, with a\colorbox{yellow}{lag time} consistent with zero. Throughout every stage of the merging, BH accretion increases when the specific angular momentum of the central gas has a negative temporal gradient.
\par  In varying the initial mass ratio (from 1:1 to 1:10), orbital configuration (\colorbox{yellow}{coplanar,prograde and retrograde, and inclined}), and gas fraction in the galactic discs (30 and 60 per cent), focusing on the accretion onto and mass growth of the central BHs, and on the triggering of AGN. These results can be itemized below:
\begin{enumerate}
	\item All stages can be divided by viewing the distinguishing change of specific angular momentum in different spherical shells of gas around the secondary BH, returning to a similar state before the merger.
	\item A strong anti-correlation is found between the specific angular momentum of the central gas and the BH accretion rate. The gas that gets accreted is not the gas with low angular momentum neccesarily but the gas that loses it.
	\item While not all AGN activity is driven by the merger, the merger stage shows stronger, persisent AGN activity.
	\item The parameter that mostly affects BH accretion and AGN activity in mergers, is initial mass ratio.
	\item The secondary galaxy always responds strongly to the interaction, almost independently of the mass ratio.
	\item In minor mergers, the secondary BH grows faster (fractionally) than the primary BH. While in major mergers it is the opposite.
	\item The AGN fraction is calculated as a function of separation and find that the AGN fraction increases with decreasing separations.
\end{enumerate}
\par  Thanks to the simulations high resolution, it is confirmed that BH accretion in galaxy mergers is highly linked to how capable physical processes are in inducing gas to lose angular momentum and flow towards the center of the galaxy. An effectiveness not very dependent on the initial orbital configuration and/or on the gas fraction, yet strongly dependent on the initial mass ratio. The time evolution of the BH mass ratio, \textit{q}, allows for an opportunity to understand which BH would grow the most during a merger event. \textit{q} also changes significantly during the merger, the direction of that change is due to the initial mass ratio.

\section{Thought Process}
\
\maketitle{\textbf{\underline{\emph{Highlighted Words}}}:}
\
\par
\begin{enumerate}
	\item \textbf{hydrodynamical simulations}:	Googled this and came up with the Illustris page giving a well versed explanation about what hydrodynamical simulations are and what they do. Essentially we use hydrodynamics to explain the movement of gas in the cosmos and these hydrodynamical simulations are really powerful in recreating certain scenarios if fed the right information.
	\item \textbf{orbital configuration}: I understood what this meant but I was curious as to what this really meant and the different scenarios are possible in what they were talking about. Just googling orbital configuration spits out things about the orbits of electrons and stuff, but putting in astrophysics as another keyword I get out orbital mechanics. While looking through the search results I ran into this useful website that really carries a lot of info, \url{http://www.astronomy.marcric.com/pages/07-astrophysics_orbits.html}.
	\item \textbf{the gas fraction}: I googled this and got mostly links to arxiv papers using the word in it, but I feel like this just means the amount of stuff in the galaxy that is baryonic matter.
	\item \textbf{gas inflows}: From just the words itself I can say that it's just gas flowing inward, but is that inflow constant or is it affected in any way throughout the merging process.
	\item \textbf{secular}: after reading the paper more I figured out that this just meant internal for this paper.
	\item \textbf{BH mass contrast}: I'm not very sure what this meant and googling it didn't do much for me. But just taking the word contrast meaning different than I assume it's just the difference between the BHs. Which now makes more sense to me.
	\item \textbf{dynamically quiescent}: I later on put it together that this just meant that the system isn't active.
	\item \textbf{bars and violent gas instabilities at high redshift}: I'm not really sure what this is, googled it and most that comes out is about galactic nuclei and it's bulge. Could just mean the activity near the center perhaps.
	\item \textbf{kinematic-neighbour studies}: I know that kinematics is about movement between objects, but I wanted to understand more about stellar kinematics, mostly discussing kinematic age. Because I was interested in something about finding out the age of something using stellar kinematics. But from this paper I guess it is finding things out about the object using its interactions with its surroundings.
	\item \textbf{distortion fraction}: Is this about how things look due to gravitational lensing by these BHs or is it how the galaxy looks by how much the BH's accreting.
	\item \textbf{1:1 mergers}: Figured out later that this just means mergers with the same mass ratio to one another.
	\item \textbf{consistent inflows}: Figured it out same way as gas inflows.
	\item \textbf{gravitational torques}: I've always had a little problem with torque but I get that it is dependent on distance and force, on google it says that the net torque is centered on the center of mass, but I'd like to know some more about it.
	\item \textbf{unpertubated}: Figured it out later on, just means undisturbed basically.
	\item \textbf{the peak of the cosmic merger rate}: Does this relate to the universal merger rate or is this relating to each indivdual merger.
	\item \textbf{1:4 coplanar, prograde-prograde merger}: I get that it's a merger between a black hole and another black hole with 4 times that mass and that they both cohabit the same plane, but I don't understand the prograde part. I know that prograde is when something orbits the same direction as the rotation of the object it's orbiting. But what does that mean for a prograde prograde? That they both orbit each other as the gas accretes in the same direction?
	\item \textbf{specific angular momentum}: I knew about regular angular momentum but looking up specific angular momentum clarified better, especially since the wiki page said it plays a pivotal role for two body problems, which this paper is definitely discussing about. It's the angular momentum per mass, which correlates well for this paper as well, dealing with accretion.
	\item \textbf{quasi-periodic}: Apparently according to Merriam-Webster dictionary this means motion of a system that is periodic on a small scale yet unpredictable at some larger scale. So I'm guessing this is especially important when looking at black holes, since they're periodic depending on the mass they accrete?
	\item \textbf{orders of magnitude}: It's always a problem for me to get what this means, but it's just the exponent using base 10, so the shift between 100 and 10000 is 2 orders of magnitude, I feel like this will always give me trouble regardless of how simple it is until I get it stuck in my head.
	\item \textbf{merger dynamics}: Just interested in what more can be done when looking at mergers, like different parameters and their importance, aside from the ones mentioned here.
	\item \textbf{temporal gradient}: I know that the gradient is another way to represent partial derivatives so it's a function that represents the slope of the original function as a vector representation. But what exactly does that mean for time, in this paper does it just mean how time flows as you get closer to the center of the black hole, due to time dilation?
	\item \textbf{anti-correlated}: Basically is the opposite of correlation, but I assumed it just meant that if two things are anti-correlated then there's no relationship. But it just means that when one value increases the other decreases so it's just a negative relationship.
	\item \textbf{lag time}: So it's the amount of time between one event and another, so when the paper said the lag time was close to zero, that's an incredibly quick process. Which is quite amazing.
	\item \textbf{coplanar,prograde and retrograde, and inclined}: The coplanar, prograde and retrograde I understood while I was looking at (16) but the inclined part I was just wondering more about that, thinking about it being inclined in either the azimuthal or polar coordinates or both as well, which would make more sense to me.
\end{enumerate}
\maketitle{\textbf{\underline{\emph{While Reading}}}:}
\begin{enumerate}
	\item \textbf{stellar softening}: Is this just the point where star formation starts to decrease to normal level?
	\item \textbf{angular momentum flip}: This sounds pretty cool, the fact that the angular momentum just changes, how does that affect the system, and why does it flip?
	\item \textbf{feedback efficiency}: Is this just the effect of accretion of the black hole or is it some other factor, why are the $\epsilon{_f}$ so small, if it was larger then what?
	\item \textbf{spatial and temporal resolution}: So how is the resolution affected by changing the values of 10pc and ~1Myr, does increasing the number gives a better resolution? How does it work? 
\end{enumerate}

\section{Importance}
\
\par  I think this paper is important so that I can understand what occurs during mergers, to the gas being accreted to the BH. I also believe that it also gives good information that ties into the RG118 paper when thinking about both of them together. So that it helps me be moore knowledgeable about the certain things that could and do happen during mergers so that I can look out for signs like that, during plots perhaps. 

\end{document}